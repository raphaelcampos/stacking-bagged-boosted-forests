\documentclass[]{article}
\usepackage{lmodern}
\usepackage{amssymb,amsmath}
\usepackage{ifxetex,ifluatex}
\usepackage{fixltx2e} % provides \textsubscript
\ifnum 0\ifxetex 1\fi\ifluatex 1\fi=0 % if pdftex
  \usepackage[T1]{fontenc}
  \usepackage[utf8]{inputenc}
\else % if luatex or xelatex
  \ifxetex
    \usepackage{mathspec}
    \usepackage{xltxtra,xunicode}
  \else
    \usepackage{fontspec}
  \fi
  \defaultfontfeatures{Mapping=tex-text,Scale=MatchLowercase}
  \newcommand{\euro}{€}
\fi
% use upquote if available, for straight quotes in verbatim environments
\IfFileExists{upquote.sty}{\usepackage{upquote}}{}
% use microtype if available
\IfFileExists{microtype.sty}{%
\usepackage{microtype}
\UseMicrotypeSet[protrusion]{basicmath} % disable protrusion for tt fonts
}{}
\usepackage[margin=1in]{geometry}
\usepackage{graphicx}
\makeatletter
\def\maxwidth{\ifdim\Gin@nat@width>\linewidth\linewidth\else\Gin@nat@width\fi}
\def\maxheight{\ifdim\Gin@nat@height>\textheight\textheight\else\Gin@nat@height\fi}
\makeatother
% Scale images if necessary, so that they will not overflow the page
% margins by default, and it is still possible to overwrite the defaults
% using explicit options in \includegraphics[width, height, ...]{}
\setkeys{Gin}{width=\maxwidth,height=\maxheight,keepaspectratio}
\ifxetex
  \usepackage[setpagesize=false, % page size defined by xetex
              unicode=false, % unicode breaks when used with xetex
              xetex]{hyperref}
\else
  \usepackage[unicode=true]{hyperref}
\fi
\hypersetup{breaklinks=true,
            bookmarks=true,
            pdfauthor={Raphael Rodrigues Campos},
            pdftitle={Effectivess comparison report},
            colorlinks=true,
            citecolor=blue,
            urlcolor=blue,
            linkcolor=magenta,
            pdfborder={0 0 0}}
\urlstyle{same}  % don't use monospace font for urls
\setlength{\parindent}{0pt}
\setlength{\parskip}{6pt plus 2pt minus 1pt}
\setlength{\emergencystretch}{3em}  % prevent overfull lines
\setcounter{secnumdepth}{0}

%%% Use protect on footnotes to avoid problems with footnotes in titles
\let\rmarkdownfootnote\footnote%
\def\footnote{\protect\rmarkdownfootnote}

%%% Change title format to be more compact
\usepackage{titling}

% Create subtitle command for use in maketitle
\newcommand{\subtitle}[1]{
  \posttitle{
    \begin{center}\large#1\end{center}
    }
}

\setlength{\droptitle}{-2em}
  \title{Effectivess comparison report}
  \pretitle{\vspace{\droptitle}\centering\huge}
  \posttitle{\par}
  \author{Raphael Rodrigues Campos}
  \preauthor{\centering\large\emph}
  \postauthor{\par}
  \predate{\centering\large\emph}
  \postdate{\par}
  \date{January 17, 2016}

\usepackage{multirow}


\begin{document}

\maketitle


\section{Experimento}\label{experimento}

Utilizei o executável \textbf{\emph{tcpp}} compilado pelo Thiago Salles
que estava no pacote que ele enviou no último email.

Para cada um dos \emph{dataset} eu rodei \emph{cross-validation
10-folds}. Para comparaćão dos métodos foi utilizado test t com correćão
de bonferroni. Os valores em negritos representam os vencedores e são
estatisticamente significantes.

\section{Resultados}\label{resultados}

Fiz a comparaćão entre 5 métodos, são eles: Random Forest(RF), Random
Forest com 2000 árvores (RF2000), Lazy (KNN + RF), KNN e BROOF.

Os paramêtros usados para RF, RF2000 e BROOF foram os mesmo para cada
dataset (exceto o número de árvores). Para os métodos baseados no KNN
foi usado k = 30.

A tabela a seguir compara todos o métodos. Como pode-se notar o método
Lazy ganhou ou empatou com todos os métodos em todos os 4
\emph{datasets}.

\% latex table generated in R 3.2.3 by xtable 1.8-0 package \% Thu Feb
11 17:55:49 2016

\begin{table}[ht]
\centering
\begin{tabular}{llllll}
  \hline
V1 & V2 & REUTERS90 & 20NG1 & 4UNI & ACM \\ 
  \hline
\multirow{2}{*}{RF2000} & microF1 & \bf{63.08 $\pm$  2.46} & \bf{88.07 $\pm$  1.02} & \bf{81.17 $\pm$  1.16} & 71.01 $\pm$  0.88 \\ 
   & macroF1 & \bf{24.72 $\pm$  1.09} & \bf{88.14 $\pm$  0.72} & \bf{73.19 $\pm$  0.93} & \bf{60.25 $\pm$  2.18} \\ 
  \multirow{2}{*}{BROOF} & microF1 & 63.12 $\pm$  2.39 & \bf{87.82 $\pm$  1.03} & \bf{81.12 $\pm$  1.06} & 70.99 $\pm$  0.8 \\ 
   & macroF1 & \bf{24.63 $\pm$  1.22} & \bf{87.76 $\pm$  0.79} & \bf{73 $\pm$  0.82} & \bf{60.34 $\pm$  2.11} \\ 
  \multirow{2}{*}{KNN} & microF1 & \bf{65.93 $\pm$  2.66} & 55.63 $\pm$  4.38 & 48.38 $\pm$  1.29 & 66.94 $\pm$  0.56 \\ 
   & macroF1 & \bf{24.03 $\pm$  2.08} & 66.36 $\pm$  2.87 & 26.06 $\pm$  1.36 & 57.34 $\pm$  1.59 \\ 
  \multirow{2}{*}{LAZY} & microF1 & \bf{65.12 $\pm$  2.94} & \bf{88.95 $\pm$  0.62} & \bf{80.72 $\pm$  0.77} & \bf{73.69 $\pm$  0.44} \\ 
   & macroF1 & \bf{26.01 $\pm$  1.98} & \bf{88.78 $\pm$  0.54} & \bf{72.01 $\pm$  0.9} & \bf{63.63 $\pm$  1.19} \\ 
  \multirow{2}{*}{RF} & microF1 & 63.11 $\pm$  2.41 & 86.84 $\pm$  1.06 & \bf{80.87 $\pm$  1.5} & 70.61 $\pm$  0.77 \\ 
   & macroF1 & \bf{24.79 $\pm$  1.7} & 86.77 $\pm$  0.74 & \bf{72.78 $\pm$  1.73} & \bf{60.39 $\pm$  1.45} \\ 
   \hline
\end{tabular}
\caption{Comparaćão entre todos os métodos} 
\end{table}

A tabela a seguir compara somente RF, RF2000 e BROOF, pois eu estava
achando que o BROOF da implementaćão que o Thiago me passou não era nada
mais que uma RF com muitas árvores (por isso a comparacão com uma RF de
2000 árvores). E como pode-se notar na tabela abaixo, os métodos tiveram
empate estatístico em todos os datasets.

\% latex table generated in R 3.2.3 by xtable 1.8-0 package \% Wed Feb
10 22:07:58 2016

\begin{table}[ht]
\centering
\begin{tabular}{llllll}
  \hline
V1 & V2 & REUTERS90 & 20NG1 & 4UNI & ACM \\ 
  \hline
\multirow{2}{*}{RF2000} & microF1 & \bf{63.08 $\pm$  2.46} & \bf{88.07 $\pm$  1.02} & \bf{81.17 $\pm$  1.16} & \bf{71.01 $\pm$  0.88} \\ 
   & macroF1 & \bf{24.72 $\pm$  1.09} & \bf{88.14 $\pm$  0.72} & \bf{73.19 $\pm$  0.93} & \bf{60.25 $\pm$  2.18} \\ 
  \multirow{2}{*}{BROOF} & microF1 & \bf{63.12 $\pm$  2.39} & \bf{87.82 $\pm$  1.03} & \bf{81.12 $\pm$  1.06} & \bf{70.99 $\pm$  0.8} \\ 
   & macroF1 & \bf{24.63 $\pm$  1.22} & \bf{87.76 $\pm$  0.79} & \bf{73 $\pm$  0.82} & \bf{60.34 $\pm$  2.11} \\ 
  \multirow{2}{*}{RF} & microF1 & \bf{63.11 $\pm$  2.41} & \bf{86.84 $\pm$  1.06} & \bf{80.87 $\pm$  1.5} & 70.61 $\pm$  0.77 \\ 
   & macroF1 & \bf{24.79 $\pm$  1.7} & \bf{86.77 $\pm$  0.74} & \bf{72.78 $\pm$  1.73} & \bf{60.39 $\pm$  1.45} \\ 
   \hline
\end{tabular}
\caption{Comparacão entre BROOF, RF e RF2000} 
\end{table}

\end{document}
