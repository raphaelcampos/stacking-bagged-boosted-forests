\documentclass[]{article}
\usepackage{lmodern}
\usepackage{amssymb,amsmath}
\usepackage{ifxetex,ifluatex}
\usepackage{fixltx2e} % provides \textsubscript
\ifnum 0\ifxetex 1\fi\ifluatex 1\fi=0 % if pdftex
  \usepackage[T1]{fontenc}
  \usepackage[utf8]{inputenc}
\else % if luatex or xelatex
  \ifxetex
    \usepackage{mathspec}
    \usepackage{xltxtra,xunicode}
  \else
    \usepackage{fontspec}
  \fi
  \defaultfontfeatures{Mapping=tex-text,Scale=MatchLowercase}
  \newcommand{\euro}{€}
\fi
% use upquote if available, for straight quotes in verbatim environments
\IfFileExists{upquote.sty}{\usepackage{upquote}}{}
% use microtype if available
\IfFileExists{microtype.sty}{%
\usepackage{microtype}
\UseMicrotypeSet[protrusion]{basicmath} % disable protrusion for tt fonts
}{}
\usepackage[margin=1in]{geometry}
\usepackage{graphicx}
\makeatletter
\def\maxwidth{\ifdim\Gin@nat@width>\linewidth\linewidth\else\Gin@nat@width\fi}
\def\maxheight{\ifdim\Gin@nat@height>\textheight\textheight\else\Gin@nat@height\fi}
\makeatother
% Scale images if necessary, so that they will not overflow the page
% margins by default, and it is still possible to overwrite the defaults
% using explicit options in \includegraphics[width, height, ...]{}
\setkeys{Gin}{width=\maxwidth,height=\maxheight,keepaspectratio}
\ifxetex
  \usepackage[setpagesize=false, % page size defined by xetex
              unicode=false, % unicode breaks when used with xetex
              xetex]{hyperref}
\else
  \usepackage[unicode=true]{hyperref}
\fi
\hypersetup{breaklinks=true,
            bookmarks=true,
            pdfauthor={Raphael Rodrigues Campos},
            pdftitle={Effectivess comparison report},
            colorlinks=true,
            citecolor=blue,
            urlcolor=blue,
            linkcolor=magenta,
            pdfborder={0 0 0}}
\urlstyle{same}  % don't use monospace font for urls
\setlength{\parindent}{0pt}
\setlength{\parskip}{6pt plus 2pt minus 1pt}
\setlength{\emergencystretch}{3em}  % prevent overfull lines
\setcounter{secnumdepth}{0}

%%% Use protect on footnotes to avoid problems with footnotes in titles
\let\rmarkdownfootnote\footnote%
\def\footnote{\protect\rmarkdownfootnote}

%%% Change title format to be more compact
\usepackage{titling}

% Create subtitle command for use in maketitle
\newcommand{\subtitle}[1]{
  \posttitle{
    \begin{center}\large#1\end{center}
    }
}

\setlength{\droptitle}{-2em}
  \title{Effectivess comparison report}
  \pretitle{\vspace{\droptitle}\centering\huge}
  \posttitle{\par}
  \author{Raphael Rodrigues Campos}
  \preauthor{\centering\large\emph}
  \postauthor{\par}
  \predate{\centering\large\emph}
  \postdate{\par}
  \date{January 17, 2016}

\usepackage{multirow}


\begin{document}

\maketitle


\section{Experimento}\label{experimento}

Utilizei o executável \textbf{\emph{tcpp}} compilado pelo Thiago Salles
que estava no pacote que ele enviou no último email.

Para cada um dos \emph{dataset} eu rodei \emph{cross-validation
10-folds}. Para comparaćão dos métodos foi utilizado test t com correćão
de bonferroni. Os valores em negritos representam os vencedores e são
estatisticamente significantes.

\section{Resultados}\label{resultados}

\% latex table generated in R 3.2.3 by xtable 1.8-0 package \% Mon Feb
22 14:34:42 2016

\begin{table}[ht]
\centering
\begin{tabular}{llllll}
  \hline
V1 & V2 & REUTERS90 & 20NG & 4UNI & ACM \\ 
  \hline
\multirow{2}{*}{RF2000} & microF1 & 63.02 $\pm$  0.79 & 0 $\pm$  0 & 81.16 $\pm$  0.88 & 71.01 $\pm$  0.36 \\ 
   & macroF1 & 24.79 $\pm$  1.38 & 0 $\pm$  0 & \bf{73.03 $\pm$  2.25} & \bf{60.55 $\pm$  1.48} \\ 
  \multirow{2}{*}{BROOF} & microF1 & \bf{66.82 $\pm$  0.7} & \bf{89.82 $\pm$  0} & \bf{83.19 $\pm$  0.57} & \bf{73.24 $\pm$  0.73} \\ 
   & macroF1 & \bf{28.53 $\pm$  0.37} & \bf{89.55 $\pm$  0} & \bf{75.22 $\pm$  1.85} & \bf{62.58 $\pm$  1.73} \\ 
  \multirow{2}{*}{KNN} & microF1 & \bf{67.31 $\pm$  0.87} & 84.47 $\pm$  0 & 73.09 $\pm$  1.04 & 69.55 $\pm$  0.6 \\ 
   & macroF1 & 25.71 $\pm$  1.2 & 83.88 $\pm$  0 & 60.44 $\pm$  1.17 & 57.63 $\pm$  1.67 \\ 
  \multirow{2}{*}{LAZY} & microF1 & 64.73 $\pm$  0.61 & 89.05 $\pm$  0 & 79.64 $\pm$  0.79 & \bf{73.22 $\pm$  0.62} \\ 
   & macroF1 & 25.3 $\pm$  0.7 & 88.74 $\pm$  0 & 69.11 $\pm$  2.63 & \bf{61.9 $\pm$  2.38} \\ 
  \multirow{2}{*}{RF} & microF1 & 63.02 $\pm$  1.06 & 87.59 $\pm$  0 & 80.16 $\pm$  0.95 & 70.67 $\pm$  0.46 \\ 
   & macroF1 & 24.63 $\pm$  1.75 & 87.26 $\pm$  0 & 70.74 $\pm$  2.67 & \bf{60.46 $\pm$  1.46} \\ 
   \hline
\end{tabular}
\caption{Comparaćão entre todos os métodos} 
\end{table}

\end{document}
